\usepackage[top=2cm, left=3cm, bottom=2cm, right=2cm, paper=letterpaper]{geometry}

\begin{flushright}
\small
\begin{tabular}{l}
RESUMEN DE LA MEMORIA\\
PARA OPTAR AL T�TULO DE \\
INGENIERO CIVIL\\
POR: RODRIGO P�REZ ILLANES \\
FECHA: DICIEMBRE 2014\\
PROF. GU�A: Sr. ALBERTO DE LA FUENTE\\
\end{tabular}
\end{flushright}
\begin{center}
RESPUESTA HIDROL�GICA DEL R�O CAUT�N EN LA IX REGI�N DE LA ARAUCAN�A, CHILE, ANTE ESCENARIOS DE CAMBIO GLOBAL. \\
\end{center}
El presente trabajo de t�tulo consisti� en la aplicaci�n de un modelo hidrol�gico \citep{mendoza}, para la cuenca del r�o Caut�n en la IX regi�n de la Araucan�a, en escenarios de cambio global, tales como cambios de uso de suelo y cambio clim�tico para un horizonte de 50 a�os (2015 - 2064). Se construyeron distintos escenarios de uso de suelo, bajo escenarios de cambio clim�tico. Para esto se utiliz� un modelo de la cuenca disponible en el Departamento de Ingenier�a Civil e integr� la incertidumbre asociada a los escenarios de cambio global.\\

El objetivo general fue estudiar el efecto del cambio clim�tico y de uso de suelo sobre la respuesta hidrol�gica de la cuenca del r�o Caut�n de Chile, con �nfasis en la magnitud y recurrencia de eventos extremos. Para esto se analizaron los posibles escenarios de cambio global para la regi�n de La Araucan�a, en t�rminos de clima y uso de suelos. Luego se determin� la magnitud y recurrencia de sequ�as en distintos escenarios futuros y se determin� la vulnerabilidad de la poblaci�n de la comuna Padre las Casas, ante crecidas del r�o Caut�n.\\

%Los resultados obtenidos incluyeron:\\

%\begin{itemize}
%\item Una importante disminuci�n en la recurrencia de caudales m�ximos.
%\item Un aumento en la frecuencia de la sequ�a tanto hidrol�gica como meteorol�gica, debido a la disminuci�n de la disponibilidad del recurso h�drico.
%\item Vulnerabilidad nula de la poblaci�n de la comuna Padre las Casas ante crecidas debido a la disminuci�n de los caudales m�ximos.
%\end{itemize} 

\pagebreak
